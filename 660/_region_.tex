\message{ !name(notes.tex)}\documentclass[a4paper]{article}
\usepackage{amsmath}
\begin{document}

\message{ !name(notes.tex) !offset(-3) }

\title{Scientific Computing CS660 Fall '11}
\author{Angjoo Kanazawa}
\maketitle
\section{September 1st Class one}
\subsection{Logistics}
\label{sec:logistics}
Prof. Howard Elman CSI 2120 TR 2pm-3:15pm
class url:http://www.cs.umd.edu/~elman/660.11/syl.html
\begin{itemize}
\item Scientific Computing puts heavier emphasis on computing, Numerical
  Analysis is more about proofs/theories.

\item 4-6 hw asg: \textbf{35\%} Penalty on late assignments (-15\%
  after 24 hrs, -30\% after 48 hrs. 
\item in-class midterm: \textbf{25\%}
\item final project: \textbf{40\%}
\end{itemize}

\subsection{Content} 
\label{sec:content}
\textbf{Newton's Method}: Root finding. 
Objective: Fine $x$ s.t. $f(x) = 0$, where $f$ is a scalar,
$f:\mathbf{R}\Rightarrow \mathbf{R}$
function. Where does the function cross 0 (x-axis)?

Given $x_n$, some guess, find where the line through $(x_n, f(x_n))$
tangent to the solution curve intersects the $x$-axis. Call that
pt of intersection $x_{n+1}$

The equation of the tangent line: $$\frac{y-f(x_n)}{x-x_n} = f'(x_n)$$
Set $y=0$: then
\begin{align*}  \frac{0-f(x_n)}{x_{n+1}-x_n} &= f'(x_n)\\
  \frac{x_{n+1}-x_n}{-f(x_n)} &= 1/f'(x_n)\\
  x_{n+1} &= x_n - f(x_n)/f'(x_n)\\
\end{align*}

\textbf{Another Derivation}
Consider the taylor series $f(x_n+(x-x_n)) = f(x_n) + f'(x_n)(x-x_n) +
1/2f''(x_n)(x-x_n)^2 + \text{ etc}$. This is a function of some
variable $x$. Approximate it just by using the first two terms (linear
approximation).
So above becomes a new function $f(x_n+(x-x_n))~= f(x_n) +
f'(x_n)(x-x_n) = l(x)$.
Find where $l(x) = 0$. That is: $x_{n+1} = x_n - f(x_n)/f'(x_n)$

This is not guaranteed to work, i.e. when the tangent doesn't cross
the $x$-axis.


\textbf{Problem}: given $\alpha \in \mathbf{R} > 0$ Find $1/\alpha$ without
doing any division.
First thing you need to do if identify (concoct) a $f(x)$ whose root
is $1/\alpha$.
Naive: try $f(x) = x - 1/\alpha$. But this won't work, because this
requires division.
Howabout: $f(x) = \alpha x - 1$. $f'(x) = \alpha$, so in the newton
iteration.. 
\begin{align*}
x_{n+1} &= x_n - \frac{\alpha x - 1}{\alpha}\\
&= x_n - x_n + 1/\alpha \\
&= 1/\alpha
\end{align*}
No! because you need to divide.

Answer: $f(x) = \alpha - 1/x$. Not transparent, because intuitively it
looks like there's a divide into it.
$f'(x) = - -1/x^2 = 1/x^2$. Then, $f/f' = \alpha x^2 - x$. So given $x_n$,
the iteration goes $x_{n+1} = x_n - (\alpha x_n^2 - x_n) = 2x_n -
\alpha x_n^2 = x_n(2-\alpha x_n)$.

Numerical example in matlab: let $\alpha = 2$, solve with this method. Use $x_0
= 0.1$. Notice that $err(i)/err(i-1)^2$ is constant and \emph{is} $\alpha$.
Notice the $err(i)$ decreases faster as iteration moves on, this is
the super linear convergence property of Newton's method.
With $\alpha = 0.25$, same thing.\\

\textbf{Analysis of the Newton's Method}:
$|x_{x_{n+1}}|/|x-x_n|^2 \approx 1/2 | f''(x_n)/f'(x_n)| \approx 1/x$
So @ $x = 1/\alpha$, this turns out to be $\alpha$, just for this
example.

The ratio was derived from the idea to find a patter in the errors
s.t. $e_{n+1} ~ c e_n^2$, for some $c\in \mathbf{R}$

The question is to find trends in data, and this relationship $e_{n+1}
~ ce_n^p$ is useful to tell us the rate of convergence as the solution
approches the optimal one. (I think $p$ goes down to the golden
ratio). Our goal is to find what $p$ is for a specific problem.

\section{September 6th class 2}
\label{sec:class2}

\subsection{Process of Scientific Computing}

\begin{itemize}
\item Start with a mathematical model. (In general we don't have an analytic solution, so we get
  insight from numerical computation)
\item Example with heat conduction in a bar:
  \begin{itemize}
  \item A 1-D object $\in [0, 1]$, $u(x)=$temperature in the bar, with
  $u(0)=0, u(1) = 0$. $q$=heat flow induced by a heater of intensity $f$.
  \item We want what the temperature will be given $q$.
  \item get some models: \emph{Fourrier's Law}: $q= -ku'$, $k$=conductivity
    coefficient, transfer of heat in direction of decreasing
    temperature (hence the -). \emph{Conservation of energy}: $q' = f$.
  \item \textbf{Goal}: find $u$. \textbf{Equation of interest}: $-(ku')' = f$, the 1D diffusion
    equation.
  \item Typical strategy: lay down a grid $x_o=0, x_1, \dots, x_n,
    x_n+1=1$. Compute a discrete solution, $\bar{u}$, vector of size $n$. $\bar{u} = [u_1, \cdot, u_n]^T$, where $u_i \approx u(x_i)$
  \item \textbf{Claim}: we can find the discrete sol, $\bar{u}$ by
    solving an algebraic system of equations. For this example, this
    system is a matrix equation $$A\bar{u} = \bar{b}$$

  \end{itemize}
  \item No matter how hard I try, we're never going to get the exact
    solution. This process $A\bar{u}=\bar{b}$ leads to errors
  \item \textbf{Sources of error}:
    \begin{enumerate}
       \item Modeling error: we may not know $k$ exactly.
       \item Discretization error/Truncation error: difference between the discrete and
         the continuous values (from the approximation on a discrete
         set of points)
       \item Representation error: we don't have the entire
         $\mathbf{R}$, we only have a finite set, in floating point
         format. ($A$ and $b$ may have error)
       \item Additional error: from the computation of $\tilde{u}$,
         will get something else $\hat{\tilde{u}}\neq \tilde{u}$

         
    \end{enumerate}
  \item We're really trying to solve $\tilde{A}\tilde{u} =
    \tilde{b}$., where $\tilde{A}\approx A$, $\tilde{u} =
    \bar{u}$. Typically $\approx$ is machine precision, $10^{-16}$

  \item In the end, we want $u(x)$, and would be happy with $u(x_j)$,
    $j=1,\dots, n$. That will get $\tilde{u_j}$

\end{itemize}



\end{document}

\message{ !name(notes.tex) !offset(-150) }
